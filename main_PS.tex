\documentclass[a4paper]{article}
\usepackage{pgf,tikz,pgfplots}
\usetikzlibrary{arrows,decorations.markings}
\pgfplotsset{compat=1.15}
\usepackage{mathrsfs}
\usetikzlibrary{arrows}
%% Language and font encodings
\usepackage[english]{babel}
\usepackage[utf8x]{inputenc}
\usepackage[T1]{fontenc}
\usepackage{float}
%% Sets page size and margins
\usepackage[a4paper,top=3cm,bottom=2cm,left=3cm,right=3cm,marginparwidth=1.75cm]{geometry}
\usepackage{fancyhdr}
\pagestyle{fancy}
%% Useful packages

\usepackage{amsmath}
\usepackage{amsthm}
\usepackage{enumitem}
\usepackage{eqnarray}
\usepackage{float}
\usepackage{esint}
\usepackage{wrapfig}
\usepackage{gensymb}
\usepackage{lipsum}
\usepackage{amssymb}
\usepackage{array}
\usepackage{tikz}
\usepackage[colorlinks=true, allcolors=blue]{hyperref}
\usepackage{graphicx}
\usepackage{amsmath}
\usepackage{amssymb}
\usepackage{graphicx}
\usepackage[colorlinks=true, allcolors=blue]{hyperref}
\usepackage{mathtools}
\DeclareMathOperator{\Proj}{Proj}
\DeclareMathOperator{\lcm}{lcm}
\DeclareMathOperator{\cosec}{cosec}
\DeclareMathOperator{\sgn}{sgn}
\DeclareMathOperator{\Span}{span}
\DeclareMathOperator{\nullity}{nullity}
\DeclarePairedDelimiter\floor{\lfloor}{\rfloor}
\DeclareMathOperator{\Res}{Res}
\DeclareMathOperator{\rank}{rank}
\DeclareMathOperator{\Ker}{Ker}
\DeclareMathOperator{\R}{R}
\DeclareMathOperator{\Tr}{Tr}
\DeclareMathOperator{\diag}{diag}
\DeclareMathOperator{\Log}{Log}
\DeclareMathOperator{\sech}{sech}
\DeclareMathOperator{\Var}{Var}
\newtheorem{ans}{Answer}[section]

\definecolor{darkblue}{RGB}{	0, 0, 139}
\newtheoremstyle{new}% <name>
{2pt}% <Space above>
{2pt}% <Space below>
{\color{darkblue}}% Body font
{}% <Indent amount>
{\bfseries\color{black}}% Theorem head font
{:}% <Punctuation after theorem head>
{.5em}% <Space after theorem headi>
{}% <Theorem head spec (can be left empty, meaning `normal')>
\theoremstyle{new}
\newtheorem{qns}{Problem}[section]
\setlength{\parindent}{0cm}
\title{\textbf{Part II TSP Problem Sheet Solutions}}
\author{Tai Yingzhe, Tommy (ytt26)}
\date{}
\setlength{\parindent}{0cm}
\begin{document}
\maketitle
\tableofcontents
\newpage
\section{Problem Sheet 1}
\subsection*{Thermodynamics}
\begin{qns}[Van der Waals gas]
Show that, for a van der Waals gas, the specific heat at constant volume, $C_V$ , obeys 
$$\bigg(\frac{\partial C_V}{\partial V}\bigg)_T=0$$
\end{qns}
\begin{ans}
The heat capacity at constant volume is defined to be
$$C_V=\bigg(\frac{\partial U}{\partial T}\bigg)_V=T\bigg(\frac{\partial S}{\partial T}\bigg)_V$$
Take the partial derivative
$$\bigg(\frac{\partial C_V}{\partial V}\bigg)_T=T\bigg(\frac{\partial}{\partial T}\bigg)_V\bigg(\frac{\partial S}{\partial V}\bigg)_T=T\bigg(\frac{\partial^2p}{\partial T^2}\bigg)_V$$
where we used the Maxwell relation $(\frac{\partial S}{\partial V})_T=(\frac{\partial p}{\partial T})_V$. But the van der Waals equation of state is
$$p=\frac{Nk_BT}{V-Nb}-\frac{N^2a}{V^2}\implies\bigg(\frac{\partial p}{\partial T}\bigg)_V=\frac{Nk_B}{V-Nb}\implies\bigg(\frac{\partial C_V}{\partial V}\bigg)_T=T\bigg(\frac{\partial^2p}{\partial T^2}\bigg)_V=0$$
\end{ans}
\begin{qns}[Potentials and thermodynamic variables]
The Gibbs free energy of an imperfect gas containing $N$ molecules is given, in terms of its natural variables $T$, $p$ and $N$, by
$$G=Nk_BT\ln\frac{p}{p_0}-NA(T)p$$
where $p_0$ is a constant and $A$ is a function of $T$ only. Derive expressions in terms of $T$, $p$, $V$, and $N$ for:
\begin{enumerate}[label=(\alph*)]
\item the equation of state of the gas;
\item the entropy, $S$;
\item the enthalpy, $H$;
\item the internal energy, $U$;
\item the Helmholtz free energy, $F$.
\end{enumerate}
Can all equilibrium thermodynamic information about the gas be obtained from a knowledge of: (i) $F(T, V, N)$; (ii) the equation of state and $U(T, p, N)$?
\end{qns}
\begin{ans}
Note that $dG=-SdT+Vdp$.
\begin{enumerate}[label=(\alph*)]
\item To find the equation of state (function of $p$, $T$ and $V$), we have
$$V=\bigg(\frac{\partial G}{\partial p}\bigg)_{T,N}=\frac{Nk_BT}{p}-NA$$
\item The entropy is a conjugate variable with temperature
$$S=-\bigg(\frac{\partial G}{\partial T}\bigg)_{p,N}=-Nk_B\ln\frac{p}{p_0}+Np\bigg(\frac{\partial A}{\partial T}\bigg)_{p,N}$$
\item By definition, the enthalpy is
$$H=G+TS=Nk_BT\ln\frac{p}{p_0}-NA(T)p+T\bigg\{-Nk_B\ln\frac{p}{p_0}+Np\bigg(\frac{\partial A}{\partial T}\bigg)_{p,N}\bigg\}=Np\bigg(\bigg(\frac{\partial A}{\partial T}\bigg)_{p,N}T-A\bigg)$$
\item By definition, the internal energy is
$$U=H-pV=Np\bigg(\bigg(\frac{\partial A}{\partial T}\bigg)_{p,N}T-A\bigg)-(Nk_BT-NpA)=TN\bigg[p\bigg(\frac{\partial A}{\partial T}\bigg)_{p,N}-k_B\bigg]$$
\item By definition, the Helmholtz free energy is
$$F=U-TS=TN\bigg[p\bigg(\frac{\partial A}{\partial T}\bigg)_{p,N}-k_B\bigg]+TNk_B\ln\frac{p}{p_0}-NpT\bigg(\frac{\partial A}{\partial T}\bigg)_{p,N}=Nk_BT\bigg(\ln\frac{p}{p_0}-1\bigg)$$
\end{enumerate}
\begin{enumerate}[label=\roman*]
\item We have the free energy to be
$$dF=dU-d(TS)=-SdT-pdV+\mu dN$$
So, $T$, $V$ and $N$ are natural variables of $F$. Hence, all equilibrium thermodynamic information can be obtained.
\item Internal energy is however, $dU=TdS-pdV+\mu dN$ and so the natural variables of $U$ are $S$, $V$ and $N$. We need to change $T\rightarrow S$ and $p\rightarrow V$. But, the equation of state only allow one such change.
\end{enumerate}
\end{ans}
\begin{qns}[Entropy of the monatomic gas]
The entropy of a monatomic ideal gas is given by the Sackur-Tetrode equation which can be written in the form:
$$S(U,V,N)=Nk_B\ln\bigg\{\alpha\frac{V}{N}\bigg(\frac{U}{N}\bigg)^{3/2}\bigg\}$$
where $\alpha$ is a constant to be derived later in the course. Invert this expression to get $U(S, V, N)$. From this, obtain the equation of state expressing $p$ as a function of $V$, $N$ and $T$.
\end{qns}
\begin{ans}
Invert the expression:
$$S(U,V,N)=\frac{3}{2}Nk_B\ln\bigg\{\bigg(\alpha\frac{V}{N}\bigg)^{2/3}\frac{U}{N}\bigg\}\implies U(S,V,N)=e^{2S/3Nk_B}\bigg(\frac{N}{\alpha V}\bigg)^{2/3}N$$
But we have $dU=TdS-pdV+\mu dN$ and so
$$p=-\bigg(\frac{\partial U}{\partial V}\bigg)_{S,N}=\frac{2U}{3V},\quad T=\bigg(\frac{\partial U}{\partial S}\bigg)_{V,N}=\frac{2}{3Nk_B}U$$
which separately gives 
$$p=\frac{2}{3}U/V,\quad U=\frac{3}{2}Nk_BT$$
The latter is consistent with the equipartition of energy while the former
$$p=\frac{2}{3}\frac{3}{2}\frac{N}{V}k_BT\implies Nk_BT=pV$$
is consistent with the ideal gas law.
\end{ans}
\newpage
\begin{qns}[Analytic thermodynamics]
Use a Maxwell relation and the chain rule to show that for any substance the rate of change of $T$ with $p$ in a reversible adiabatic compression is given by
$$\bigg(\frac{\partial T}{\partial p}\bigg)_S=\frac{T}{C_p}\bigg(\frac{\partial V}{\partial T}\bigg)_p$$
Find an equivalent expression for the adiabatic rate of change of $T$ with $V$ , and check that both results are valid for an ideal monatomic gas.
\end{qns}
\begin{ans}
We have $dH=TdS+Vdp$, which gives the Maxwell relation
$$\bigg(\frac{\partial T}{\partial p}\bigg)_S=\bigg(\frac{\partial V}{\partial S}\bigg)_p=\bigg(\frac{\partial V}{\partial T}\bigg)_p\bigg(\frac{\partial T}{\partial S}\bigg)_p$$
But, $C_p=(\frac{\partial U}{\partial T})_p=T(\frac{\partial S}{\partial T})_p$, which gives our relation. Next, we want to find $(\frac{\partial T}{\partial V})_S$. We have $dU=TdS-pdV$, which gives the Maxwell relation
$$\bigg(\frac{\partial T}{\partial V}\bigg)_S=-\bigg(\frac{\partial p}{\partial S}\bigg)_V=-\bigg(\frac{\partial p}{\partial T}\bigg)_V\bigg(\frac{\partial T}{\partial S}\bigg)_V=-\frac{T}{C_V}\bigg(\frac{\partial p}{\partial T}\bigg)_V$$
where $C_V=(\frac{\partial U}{\partial T})_V=T(\frac{\partial S}{\partial T})_V$. For ideal gas, the equation of state is $pV=Nk_BT$. For an adiabatic process, $pV^\gamma=\text{const.}$ we have
$$0=d(pV^\gamma)=d[(Nk_BT)^\gamma p^{1-\gamma}]=\gamma T^{\gamma-1}p^{1-\gamma}dT+(1-\gamma)T^\gamma p^{-\gamma}dp$$
For monatomic gas, we have $\gamma=5/3$. We show:
$$\bigg(\frac{\partial T}{\partial p}\bigg)_S=-\frac{1-\gamma}{\gamma}\frac{T}{p}=\frac{2T}{5p},\quad \frac{T}{C_p}\bigg(\frac{\partial V}{\partial T}\bigg)_p=\frac{T}{2.5Nk_B}\frac{Nk_B}{p}=\frac{2T}{5p}$$
We can also write the adiabatic process as
$$0=d(pV^\gamma)=V^{\gamma-1}dT+TV^{\gamma-2}(\gamma-1)dV$$
which gives
$$\bigg(\frac{\partial T}{\partial V}\bigg)_S=-(\gamma-1)\frac{T}{V}=-\frac{2T}{3V},\quad -\frac{T}{C_p}\bigg(\frac{\partial p}{\partial T}\bigg)_V=-\frac{T}{1.5Nk_B}\frac{Nk_B}{V}=-\frac{2T}{3V}$$
\end{ans}
\begin{qns}[Brief Notes]
Write brief notes on thermodynamic equilibrium in closed and open systems.
\end{qns}
\begin{ans}
The thermodynamic equilibrium state is defined as the one macroscopic state of a system which is automatically attained after a sufficiently long period of time such that the macroscopic properties of the system no longer change with time. The governing principle of thermodynamic equilibrium is that the entropy of the Universe tends to a maximum.\\[5pt]
Closed systems does not exchange energy or particles with the environment. The entropy is maximized with respect to the internal constraints of the system. Consider a system of total energy $U$, total volume $V$ and total particle number $N$ with an imaginary wall that partitions the system into two regions of volumes $V_1$, $V_2$, energies $U_1$, $U_2$ and particle numbers $N_1$, $N_2$. Since the system is closed, we have
$$V_1+V_2=V,\quad U_1+U_2=U,\quad N_1+N_2=N$$
Maximize the total entropy $S=S_1+S_2$:
\begin{align}
    0&=dS_1+dS_2\nonumber\\&=\bigg(\frac{\partial S_1}{\partial U_1}\bigg)_{V_1,N_1}dU_1+\bigg(\frac{\partial S_2}{\partial U_2}\bigg)_{V_2,N_2}dU_2+\bigg(\frac{\partial S_1}{\partial V_1}\bigg)_{U_1,N_1}dV_1\nonumber\\&+\bigg(\frac{\partial S_2}{\partial V_2}\bigg)_{U_2,N_2}dV_2+\bigg(\frac{\partial S_1}{\partial N_1}\bigg)_{V_1,U_1}dN_1+\bigg(\frac{\partial S_2}{\partial N_2}\bigg)_{V_2,U_2}dN_1\nonumber\\&=\bigg(\frac{1}{T_1}-\frac{1}{T_2}\bigg)dU+\bigg(\frac{P_1}{T_1}-\frac{P_2}{T_2}\bigg)dV_1-\bigg(\frac{\mu_1}{T_1}-\frac{\mu_2}{T_2}\bigg)dN_1\nonumber
\end{align}
There are thus three types of equilibrium for a closed system:
\begin{itemize}
    \item thermal equilibrium: the temperatures of the two regions are the same. For constant volume and particle numbers, the system will evolve to this state by letting heat flow from the hotter region to the colder region.
    \item mechanical equilibrium: the pressures of the two regions are the same. For same temperature and particle numbers, the system will evolve to this state by moving the partition such that the higher pressure region expands into the other.
    \item chemical equilibrium: the chemical potential of the two regions are the same. For same temperature and volumes, the system will evolve to this state by allowing particles to flow down the chemical potential gradient.
\end{itemize}
We extend our arguments to an open system. Open systems allow heat and particle number exchange with the system and reservoir. The previous discussion on equilibrium conditions for a closed system can hold by treating the combined system and the reservoir as a closed system. Without loss of generality, let the labels 1 and 2 refer to the system and reservoir respectively. In another words, the system is in thermodynamic equilibrium, if it has the same temperature, particle number and volume as the reservoir. Note that the reservoir is significantly larger so its temperature, pressure and chemical potential are effectively constants.\\[5pt]
However, in maximizing the entropy of the combined system, it is not convenient to explicitly consider the entropy of the reservoir. The maximum entropy condition gives:
\begin{align}
    0&=dS_{sys}+dS_{res}\nonumber\\&=dS_{sys}+\frac{1}{T_{res}}(dU_{res}+P_{res}dV_{res}-\mu_{res}dN_{res})\nonumber\\&=\frac{1}{T_{res}}(T_{res}dS_{sys}-dU_{sys}-P_{res}dV_{sys}+\mu_{res}dN_{sys})\nonumber\\&:=-\frac{dA_{sys}}{T_{res}}\nonumber
\end{align}
where we defined the availability of the system as
$$A_{sys}:=U_{sys}-T_{res}S_{sys}+P_{res}V_{sys}-\mu_{res}N_{sys}$$
and used the conservation of energy, volume and particle numbers:
$$dU_{res}=-dU_{sys},\quad dV_{res}=-dV_{sys},\quad dN_{res}=-dN_{sys}$$
Hence, the condition of maximum entropy for both the system and the reservoir is equivalent to minimizing the availability of the system. Note that we have
$$A_{sys}=U_{sys}-T_{res}S_{sys}+P_{res}V_{sys}-\mu_{res}N_{sys}=(T_{sys}-T_{res})S_{sys}-(P_{sys}-P_{res})V_{sys}+(\mu_{sys}-\mu_{res})N_{sys}$$
So, when the system is in complete thermodynamic equilibrium with the reservoir, the availability of the system is zero. Finally, for particular constraints, minimizing the availability is equivalent to minimizing a particular thermodynamic potential. Some examples are
\begin{itemize}
    \item Helmholtz free energy $F=U-TS$: for constant $T$ and $V$
    \item Gibbs free energy $G=F-PV$: for constant $T$ and $P$
    \item Enthalpy $H=U-PV$: for constant $S$ and $P$
    \item Grand potential $\Phi=F-\mu N$: for constant $T$ and $\mu$
\end{itemize}
\end{ans}
\newpage
\begin{qns}[Bubble]
Under what conditions is the Helmholtz free energy $F$ a minimum for a system in equilibrium? The work corresponding to an increase in the surface area of a liquid is $dW = \Gamma dA$, where $\Gamma$ is the surface tension, and $A$ is the area of the surface.\\[5pt]
Consider a bubble of air in a large container of liquid in equilibrium. Write the total Helmholtz free energy of the system as the sum of contributions from the air in the bubble, $F_a$, the surface of the bubble, $F_s$, and the surrounding liquid, $F_l$. Show that the pressure of the air inside the bubble is equal to $p_l + 2\Gamma/r$, where $p_l$ is the pressure of the liquid.
\end{qns}
\begin{ans}
We have
$$dF=-SdT-pdV+\mu dN$$
so $F$ is minimum for a system at constant $T$, $V$ and $N$. Take the entire system to be the bubble and its surrounding liquid, and assume it is at fixed volume and temperature. Our goal is thus to minimize $F$ - for both bulk systems and their interface. The interface has a work contribution $-pdV=\Gamma dA$ where $\Gamma$ is the surface tension.
$$0=dF=dF_l+dF_a+dF_{\text{surface}}=p_ldV_a-p_adV_a+\Gamma dA$$
where we have $dT=0$, $dV_l=-dV_a$, $dN_i=0$ $\forall i$. For a spherical bubble, we have $dA=8\pi rdr$ and $dV_a=4\pi r^2dr$, and so
$$0=(p_l-p_a)rdr+2\Gamma dr\implies p_a=p_l+\frac{2\Gamma}{r}$$
The pressure inside the bubble is higher than the outside.
\end{ans}
\begin{qns}[Oxygen extraction]
What is the minimum work required to extract 1 mole of pure O$_2$ from a large volume of air at the same temperature and pressure, if air is regarded as being composed of 1 volume of O$_2$ mixed with 4 volumes of $N_2$. 
\end{qns}
\begin{ans}
Consider the reverse process where the pure gases are mixed. In this mixing, $\Delta W=\Delta Q=0$, so by the first law of thermodynamics, $\Delta U=0$.\\[5pt]
Consider the forward process:
$$0=\Delta U=T\Delta S+\Delta W$$
where $\Delta S<0$ is the change in entropy from separating Oxygen, and $\Delta W=-T\Delta S>0$ is the reversible work done. By Dalton's law, the initial pressure and final pressure of oxygen are $p_0/5$ and $p_0$ respectively, where $p_0$ is the atmospheric pressure. The Sackur-Tetrode equation gives
$$\Delta S=R\ln(1/p_0)-R\ln(1/(p_0/5))=-R\ln 5$$
The work done is then
$$\Delta W=-T\Delta S=T~R\ln 5=13.4~T$$
amount of Joules.
\end{ans}
\newpage
\begin{qns}[Fuel cell]
Consider the reaction
$$\text{H}_2+\text{O}_2\rightarrow\text{H}_2\text{O}$$
The following table lists parameters for this reaction at constant temperature $T = 298 K$ and pressure $p = 1 atm$.
\begin{table}[H]
\begin{tabular}{lllll}
         & H$_2$    & O$_2$    & H$_2$O (liquid)    &  \\
enthalpy $H_n^{(m)}$ (kJ/mol) & 0     & 0     & -285.8 &  \\
entropy $S_n^{(m)}$ (J/mol K) & 130.7 & 205.1 & 69.9   & 
\end{tabular}
\end{table}
We first consider a process, in which the reactants are mixed and the reaction proceeds towards equilibrium at constant pressure and temperature. Calculate the Gibbs free energies per mole, $G_n^{(m)}$ . Calculate the equilibrium constant $K_c(p,T)$. What does its value tell us about the nature of the reaction?\\[5pt]
We now consider an ideal fuel cell operating at the same pressure and temperature, in which the reaction proceeds via an electrolyte. What are the maximum possible electric work extracted and the corresponding heat produced when forming one mol of H$_2$O. What is the efficiency and the voltage of an ideal fuel cell?
\end{qns}
\begin{ans}
We have $G=H-TS$
$$G_{H_2}^{(m)}=0-298(130.7)=-38.9~kJ/mol,\quad G_{O_2}^{(m)}=0-298(205.1)=-61.1~kJ/mol$$
$$G_{H_2O}^{(m)}=-285.8\times10^3-298(69.9)=-306.6~kJ/mol$$
The chemical potential is the Gibbs free energy per particle, i.e. $\mu=G/N=N_A\mu_0$, where $\mu_0$ is the chemical potential at 1 atm and 298 K. The equilibrium constant is defined to be
$$k_BT\ln K_c(p,T)=-\sum_i\nu_i\mu_{0i}\implies K_c=\exp\bigg(-\frac{1}{RT}\bigg(2G_{H_2}^{(m)}+G_{O_2}^{(m)}-2G_{H_2O}^{(m)}\bigg)\bigg)=e^{-303.7}<<1$$
Since $K_c$ is very small, the reaction is pushed to the right, i.e. water is much favorably formed.\\[5pt]
The change in Gibbs free energy (from the formation of 1 mole of water) is the maximal reversible work that can be done by the system at the same pressure and temperature. 
$$\Delta G^{(m)}=G^{(m)}_{H_2O}-(G_{H_2}^{(m)}+\frac{1}{2}G_{O_2}^{(m)})=-306.6\times10^3-(-61.1(0.5)-38.9)\times10^3=-237~\text{kJ/mol}$$
The heat evolved from the reaction is
$$Q^{(m)}=T\Delta S^{(m)}=T(S_{H_2O}^{(m)}-S_{H_2}^{(m)}-\frac{1}{2}S_{O_2}^{(m)})=298(69.9-(130.7+0.5(205.1)))=-48.7~\text{kJ/mol}$$
If the reaction were allowed to proceed without extracting work. 
$$Q_{\text{irrev}}^{(m)}=H_{H_2O}^{(m)}-(H_{H_2}^{(m)}+0.5H_{O_2}^{(m)})=-285.58~\text{kJ/mol}$$
The efficiency of the ideal fuel cell is
$$\eta=\frac{\Delta G^{(m)}}{Q_{\text{irrev}}^{(m)}}=\frac{-237}{-285.58}=0.83$$
with the remaining 17 percent being released as heat. Assuming the fuel cell is reversible, the maximally available work will be electrical work done.
$$W'=\Delta G=QV=2N_AeV\implies-237.28\times10^3=2(6.02\times10^{23})(-1.6\times10^{-19})V\implies V=1.23V$$
where 2 moles of electron is involved.
\end{ans}
\newpage
\begin{qns}[Superconductor]
The heat capacities of the superconducting and normal phases of a metal at low temperatures are given approximately by
$$C_s(T)=V\alpha T^3\text{ superconducting phase}$$
$$C_n(T)=V\beta T^3+V\gamma T\text{ normal phase}$$
where $V$ is the volume and $\alpha$, $\beta$, and $\gamma$ are constants. Above a temperature $T_c$, the normal phase is stable. At low temperatures the superconducting phase is stable, but it can be suppressed in high magnetic fields, which enabled the experimental determination of $C_n(T)$ given above.\\[5pt]
Experiments indicate that the latent heat for the transition is zero. What does this imply for the entropy of the normal and of the superconducting phase at $T_c$? Find an expression for $T_c$.
\end{qns}
\begin{ans}
The heat capacity is $C=T\frac{dS}{dT}$, so the entropy is
$$S_{s}(T)=\int\frac{C_{s}}{T}dT=\frac{1}{3}V\alpha T^3+S^{(0)}_{s}$$
$$S_{n}(T)=\int\frac{C_{n}}{T}dT=\frac{1}{3}V\beta T^3+V\gamma T+S^{(0)}_{n}$$
The Third Law tells us that at $T=0$, $S_n(T=0)=S_s(T=0)=0$. The latent heat is given to be zero at the phase transition, so
$$0=L=T_c[S_n(T_c)-S_s(T_c)]=T_c\bigg[(\beta-\alpha)\frac{V}{3}T_c^3+V\gamma T_c\bigg]\implies T_c=\sqrt{\frac{3\gamma}{\alpha-\beta}}$$
\end{ans}
\newpage
\section{Problem Sheet 2}
\subsection*{Equilibrium thermodynamics, basic statistical physics}
\begin{qns}

\end{qns}
\begin{ans}

\end{ans}
\begin{qns}

\end{qns}
\begin{ans}

\end{ans}
\newpage
\begin{qns}

\end{qns}
\begin{ans}

\end{ans}
\begin{qns}

\end{qns}
\begin{ans}

\end{ans}
\newpage
\begin{qns}

\end{qns}
\begin{ans}

\end{ans}
\begin{qns}

\end{qns}
\begin{ans}

\end{ans}
\newpage
\begin{qns}

\end{qns}
\begin{ans}

\end{ans}
\newpage
\section{Problem Sheet 3}
\subsection*{Quantum Statistics}
\begin{qns}

\end{qns}
\begin{ans}

\end{ans}
\begin{qns}

\end{qns}
\begin{ans}

\end{ans}
\newpage
\begin{qns}

\end{qns}
\begin{ans}

\end{ans}
\begin{qns}

\end{qns}
\begin{ans}

\end{ans}
\newpage
\begin{qns}

\end{qns}
\begin{ans}

\end{ans}
\begin{qns}

\end{qns}
\begin{ans}

\end{ans}
\newpage
\begin{qns}

\end{qns}
\begin{ans}

\end{ans}
\begin{qns}

\end{qns}
\begin{ans}

\end{ans}

\newpage
\section{Problem Sheet 4}
\subsection*{Interacting systems, stochastic physics}
\begin{qns}

\end{qns}
\begin{ans}

\end{ans}
\begin{qns}

\end{qns}
\begin{ans}

\end{ans}
\newpage
\begin{qns}

\end{qns}
\begin{ans}

\end{ans}
\begin{qns}

\end{qns}
\begin{ans}

\end{ans}
\newpage
\begin{qns}

\end{qns}
\begin{ans}

\end{ans}
\begin{qns}

\end{qns}
\begin{ans}

\end{ans}
\newpage
\begin{qns}

\end{qns}
\begin{ans}

\end{ans}
\begin{qns}

\end{qns}
\begin{ans}

\end{ans}
\newpage
\begin{qns}

\end{qns}
\begin{ans}

\end{ans}
\begin{qns}

\end{qns}
\begin{ans}

\end{ans}
\begin{qns}

\end{qns}
\begin{ans}

\end{ans}
\end{document}